%%%%%%%%%%%%%%%%%%%%%%%%%%%%%%%%%%%%%%%%%
% Medium Length Professional CV
% LaTeX Template
% Version 2.0 (8/5/13)
%
% This template has been downloaded from:
% http://www.LaTeXTemplates.com
%
% Original author:
% Trey Hunner (http://www.treyhunner.com/)
%
% Important note:
% This template requires the resume.cls file to be in the same directory as the
% .tex file. The resume.cls file provides the resume style used for structuring the
% document.
%
%%%%%%%%%%%%%%%%%%%%%%%%%%%%%%%%%%%%%%%%%
% (c) 2015 Wu Lei <araleii@mail.bnu.edu.cn> http://araleii.com
%----------------------------------------------------------------------------------------
%	PACKAGES AND OTHER DOCUMENT CONFIGURATIONS
%----------------------------------------------------------------------------------------

\documentclass[UTF8]{resume} % Use the custom resume.cls style

\usepackage[left=0.75in,top=0.6in,right=0.75in,bottom=0.6in]{geometry} % Document margins
\usepackage{ctex}

\begin{document}
\setlength{\parindent}{0pt}
\newcommand{\mywebheader}{
	\begin{tabular*}{7in}{l@{\extracolsep{\fill}}r}
		\textbf{{\LARGE 吴雷}} & {}\\
		{araleiiiiii@gmail.com} & {+86 18811478025} \\
	\end{tabular*}
	\noindent\rule{\textwidth}{2pt}
	\vspace{0.01in}}

\mywebheader
%----------------------------------------------------------------------------------------
%	EDUCATION SECTION
%----------------------------------------------------------------------------------------



\setlength{\parindent}{0pt}

\begin{rSection}{教育经历}
{\bf 北京师范大学} \hfill {\em 2012 - 2016}\\
计算机科学学士(7/37,放弃保研) \\
{\bf 加州大学河滨分校} \hfill {\em Fall quarter,2015}\\
交流学生(主修计算机安全	)
\end{rSection}

%----------------------------------------------------------------------------------------
%	WORK EXPERIENCE SECTION
%----------------------------------------------------------------------------------------

\begin{rSection}{实习经历}

\begin{rSubsection}{中软国际ETC}{ {\em 2015.07}}{Web应用开发(Java,JS)}{北京}
\item 为一个会议管理系统的进行前端和后台的开发
\item 实现人员管理,会议室管理以及预订等服务 
\end{rSubsection}


\end{rSection}

%----------------------------------------------------------------------------------------
%	TECHNICAL STRENGTHS SECTION
%----------------------------------------------------------------------------------------

\begin{rSection}{专业技能}

\begin{tabular}{ @{} >{\bfseries}l }
开发: { \rm C/C++, Java, JS, MySQL, Linux} \\
兴趣: { \rm 计算机安全,反向工程, 算法, Web开发}\\
 { \rm -我对尝试新技术充满热情,并善于学习新技能}
\end{tabular}

\end{rSection}

\begin{rSection}{获奖经历}
{中软国际实习优秀学员} \hfill {\em 2015.07}\\
{美国数学建模竞赛二等奖} \hfill {\em 2015.04} \\ 
{北京师范大学程序设计竞赛一等奖} \hfill {\em 2015.04} \\ 
{2次ACM-ICPC亚洲区域赛银奖} \hfill {\em 2014.09\&2014.11} \\
{北京师范大学京师一等奖学金} \hfill {\em 2014.10} \\
{北京师范大学竞赛一等奖学金} \hfill {\em 2014.10\&2015.11} \\
{北京师范大学``京师杯"论文竞赛一等奖} \hfill {\em 2014.10} \\
{``高教社杯"全国大学生数学建模竞赛国家一等奖} \hfill {\em 2014.09}
\end{rSection}

\begin{rSection}{项目经历}
\begin{rSubsection}{基于Firestorm\&Opensim的三维模型资源库的开发}{ {\em 2013 ~ 2015}}{国家大学生创新训练项目}{C++,JS}
	\item 为Firestorm用户提供了一个可以存储和共享三维模型的在线平台以提高用户创建虚拟世界的效率和乐趣
\end{rSubsection}	
\begin{rSubsection}{BNU Sports}{ {\em 2015}}{一个方便使用的体育场馆预订系统 }{Laravel,MySQL,JS}
	\item 源码: https://github.com/Araleii/BNUSPORTS
	\item 为北师大的学生和职员设计
\end{rSubsection}
\begin{rSubsection}{IP Search System}{ {\em 2013}}{为北师大经管学院实现的一个简单IP查询系统}{Java,MySQL,JS}
	\item 地址: http://business.bnu.edu.cn:8080/ip/
	\item 源码: https://github.com/Araleii/SearchSysWithBeanPlus
	\item 集成了3个IP数据库,并提供多种查询方式
\end{rSubsection}		
\begin{rSubsection}{Sudoku Solver}{ {\em 2015}}{一个有趣快速的基于A*算法的数独求解工具}{JS}
	\item 地址: http://araleii.com/sudoku/
	\item 源码: https://github.com/Araleii/Sudoku
	\item 为人工智能课程项目设计 
\end{rSubsection}	
\end{rSection}

\begin{rSection}{论文发表}
\item[-] Wei, Yungang, Lei Wu, Bo Sun, and Xiaoming Zhu. "Improved QEM Simplification Algorithm Based on Discrete Curvature and a Sparseness Coefficient." In IT Convergence and Security (ICITCS), 2014 International Conference on, pp. 1-5. IEEE, 2014.
\item[-] 另一篇论文-``Security and Privacy in the Internet of Vehicles" 已经被国际会议 IIKI 2015接受.
\end{rSection}


\begin{rSection}{英语能力}
	{\bf CET-6}  { 543}\\
	{\bf TOEFL}  { 93}\\
\end{rSection}

%----------------------------------------------------------------------------------------
%	EXAMPLE SECTION
%----------------------------------------------------------------------------------------

%\begin{rSection}{Section Name}

%Section content\ldots

%\end{rSection}

%----------------------------------------------------------------------------------------

\end{document}
